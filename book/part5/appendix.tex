\part{常见问题和附录}
\chapter{常见问题集}

\begin{custom}{问题}
有没有办法章节用“第一章,第一节,(一)”这种?
\end{custom}

\begin{solution}
你可以修改模板中对于章节的设置,利用 ctex 宏集的 \lstinline{\zhnumber} 命令可以把计数器的数字形式转为中文。
\end{solution}


\begin{custom}{问题}
3.07 版本的 cls 的 natbib 加了numbers 编译完了没变化,群主设置了不可更改了?
\end{custom}

\begin{solution}
3.07 中在 \lstinline{gbt7714} 宏包使用时,加入了 \lstinline{authoryear} 选项,这个使得 \lstinline{natbib} 设置了 \lstinline{numbers} 也无法生效。3.08 版本中,模板增加了 \lstinline{numbers} 和 \lstinline{authoryear} 文献选项,你可以参考前文设置说明。
\end{solution}

\begin{custom}{问题}
大佬,我想把正文字体改为亮色,背景色改为黑灰色。
\end{custom}

\begin{solution}
页面颜色可以使用 \lstinline{\pagecolor} 命令设置,文本命令可以参考\href{https://tex.stackexchange.com/questions/278544/xcolor-what-is-the-equivalent-of-default-text-color}{这里}进行设置。
\end{solution}

\begin{custom}{问题}
\lstinline{! LaTeX Error: Unknown option `scheme=plain' for package `ctex'.}
\end{custom}

\begin{solution}
你用的 C\TeX{} 套装吧?这个里面的 \lstinline{ctex} 宏包已经是已经是 10 年前的了,与本模板使用的 \lstinline{ctex} 宏集有很大区别。不建议 C\TeX{} 套装了,请卸载并安装 \TeX{} Live 2019。
\end{solution}

\begin{custom}{问题}
我该使用什么版本?
\end{custom}

\begin{solution}
请务必使用\href{https://github.com/ElegantLaTeX/ElegantBook/releases}{最新正式发行版},发行版间不定期可能会有更新(修复 bug 或者改进之类),如果你在使用过程中没有遇到问题,不需要每次更新\href{https://github.com/ElegantLaTeX/ElegantBook/archive/master.zip}{最新版},但是在发行版更新之后,请尽可能使用最新版(发行版)!最新发行版可以在 Github 或者 \TeX{} Live 2019 内获取。
\end{solution}


\begin{custom}{问题}
我该使用什么编辑器?
\end{custom}

\begin{solution}
你可以使用 \TeX{} Live 2019 自带的编辑器 \TeX{}works 或者使用 \TeX{}studio,\TeX works 的自动补全,你可以参考我们的总结 \href{https://github.com/EthanDeng/texworks-autocomplete}{\TeX works 自动补全}。推荐使用 \TeX{} Live 2019 + \TeX Studio。我自己用 VS Code 和 Sublime Text,相关的配置说明,请参考 \href{https://github.com/EthanDeng/vscode-latex}{\LaTeX{} 编译环境配置:Visual Studio Code 配置简介} 和 \href{https://github.com/EthanDeng/sublime-text-latex}{Sublime Text 搭建 \LaTeX{} 编写环境}。
\end{solution}


\begin{custom}{问题}
您好,我们想用您的 ElegantBook 模板写一本书。关于机器学习的教材,希望获得您的授权,谢谢您的宝贵时间。
\end{custom}

\begin{solution}
模板的使用修改都是自由的,你们声明模板来源以及模板地址(github 地址)即可,其他未尽事宜按照开源协议 LPPL-1.3c。做好之后,如果方便的话,可以给我们一个链接,我把你们的教材放在 ElegantLaTeX 用户作品集里。
\end{solution}

\begin{custom}{问题}
我想要原来的封面!
\end{custom}

\begin{solution}
我们计划在未来版本加入封面选择,让用户可以选择旧版封面。
\end{solution}

\begin{custom}{问题}
我想修改中文字体!
\end{custom}

\begin{solution}
首先,我们{\heiti 强烈建议你不要去修改字体}!如果你一定坚持修改字体,请在 \lstinline{newtxtext} 宏包加载前加入中文字体设置(\lstinline{xeCJK} 宏包)。如果你选择自定义字体,请设置好 \lstinline{\kaishu},\lstinline{\heiti} 等命令,否则会报错。如果你看不懂我现在说的,请停止你的字体自定义行为。
\end{solution}

\begin{custom}{问题}
请问交叉引用是什么?
\end{custom}

\begin{solution}
本群和本模板适合有一定 \LaTeX{} 基础的用户使用,新手请先学习 \LaTeX{} 的基础,理解各种概念,否则你将寸步难行。
\end{solution}

\begin{custom}{问题}
定义等环境中无法使用加粗命令么?
\end{custom}

\begin{solution}
是这样的,默认中文并没加粗命令,如果你想在定义等环境中使用加粗命令,请使用 \lstinline{\heiti} 等字体命令,而不要使用 \lstinline{\textbf}。或者,你可以将 \lstinline{\textbf} 重新定义为 \lstinline{\heiti}。英文模式不存在这个问题。
\end{solution}

\begin{custom}{问题}
代码高亮环境能用其他语言吗?
\end{custom}

\begin{solution}
可以的,ElegantBook 模板用的是 \lstinline{listings} 宏包,你可以在环境之后加上语言,全局语言修改请使用 \lstinline{\lstset} 命令,更多信息请参考宏包文档。
\end{solution}


\begin{custom}{问题}
群主,什么时候出 Beamer 的模板(主题),ElegantSlide 或者 ElegantBeamer?
\end{custom}

\begin{solution}
这个问题问的人比较多,我这里给个明确的答案。由于 Beamer 中有一个很优秀的主题 \href{https://github.com/matze/mtheme}{Metropolis}。我觉得在我们找到非常好的创意之前不会发布正式的 Beamer 主题,如果你非常希望得到 Elegant\LaTeX{} “官方”的主题,请在用户 QQ 群内下载我们测试主题 PreElegantSlide(未来不一定按照这个制作)。正式版制作计划在 2020 年之后。
\end{solution}

\begin{custom}{问题}
群主好棒,想嫁!
\end{custom}

\begin{solution}
我取向正常!
\end{solution}


\nocite{*} 

\bibliography{reference}

\appendix
\chapter{基本数学工具}

本附录包括了计量经济学中用到的一些基本数学,我们扼要论述了求和算子的各种性质,研究了线性和某些非线性方程的性质,并复习了比例和百分数。我们还介绍了一些在应用计量经济学中常见的特殊函数,包括二次函数和自然对数,前 4 节只要求基本的代数技巧,第 5 节则对微分学进行了简要回顾;虽然要理解本书的大部分内容,微积分并非必需,但在一些章末附录和第 3 篇某些高深专题中,我们还是用到了微积分。

\section{求和算子与描述统计量}

\textbf{求和算子} 是用以表达多个数求和运算的一个缩略符号,它在统计学和计量经济学分析中扮演着重要作用。如果 $\{x_i: i=1, 2, \ldots, n\}$ 表示 $n$ 个数的一个序列,那么我们就把这 $n$ 个数的和写为:

\begin{equation}
\sum_{i=1}^n x_i \equiv x_1 + x_2 +\cdots + x_n
\end{equation}
