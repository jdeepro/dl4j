% !Mode:: "TeX:UTF-8"
\chapter{Java基础}

\begin{introduction}
	\item 变量和类型
	\item 输入和输出
	\item 面向对象
	\item 异常处理
	\item 多线程
\end{introduction}

相比于其它编程语言,Java的语法不算多的,很容易掌握。
本章将以最短的篇幅带大家掌握Java编程语言。
编程语言三大核心要素:数据、语句和函数。
数据要先存放在变量中,才能被程序处理,不同类型的变量可存放不同的数据;
需要遍历所有的数据的时候,就会用到循环语句;
判断是否成立,使用判断语句等。
把常用的逻辑放在一起,就可被封装成一个函数。

\section{变量和类型}
与其他静态类型语言一样,java支持的数据类型有:
\lstinline{byte、char、short、int、long、float、double}
等。

\begin{table}[!htbp] \centering \small
\begin{tabular}{|p{1cm}|p{3cm}|p{9cm}|}
\toprule
	\multicolumn{3}{|c|}{基本类型 - 范围}\\
\midrule
	byte&8位(一个字节)&$-128\sim127$\\
	short&16位(两个字节)&$-32768\sim32767$\\
	char&16位(两个字节)&$-32768\sim32767$\\
	int&32位(四个字节)&$-2147483648\sim2147483647$\\
	long&64位(八个字节)&$-9223372036854774808\sim9223372036854774807$\\
	float&32位(四个字节)&$3.402823e+38\sim1.401298e-45$\\
	double&64位(八个字节)&$1.797693e+308\sim4.9000000e-324$\\
\bottomrule
\end{tabular}
	\caption{基本数据类型}
\end{table}

\begin{definition}{变量}{var}
	$[\text{类型}]$ 变量名称, 变量名称 $=$ 初值, $\dots$;\\
	$[\text{类型}]$ 变量名称 $=$ 初值;
\end{definition}

\begin{lstlisting}[language=java]
	int n1, n2=2, n3;
	float f1 = 0.2f;
\end{lstlisting}

\remark{
	变量要在初始化之后,才能使用。代码中小数默认是double类型,需要加后缀f才代表float。
}

\bigskip
当数据与类型不一致时,就会发生数据类型转换。
Java语言提供从小范围到大范围的自动转换,但反过来必须显式强制转换。

\begin{lstlisting}[language=java]
	byte b = 100; // 超过127会提示错误!编译器识别为int
	int n = b;
	float f = 0.2f;
	double d = f;

	int n2 = 100;
	byte b2 = (byte)n2
	double d2 = 0.2; // 小数默认是double类型
	float f2 = (float)d2;
\end{lstlisting}

\subsection{数组}
表示很多个某一类数据的时候,就要用到数组,譬如某地区的房价。
通过下标使用数组中的值,第一个下标为0,依次往后递增,最后一个下标是length-1。
Java数组下标只能是整数,不支持字符串等其他类型。
房价数据保留小数点后2-3位精度就足够了,定义float类型数组如下:

\begin{definition}{数组}{array}
	$[\text{类型}]\quad[]$变量名称;\\
	$[\text{类型}]\quad[]$变量名称 = \{131.2, 110.8, 117.34\};\\
	$[\text{类型}]\quad[]$变量名称 = new 类型$[$总数$]$\};\\
	$[\text{类型}]\quad[]$变量名称 = new 类型$[]$\{131.2, 110.8\};
\end{definition}

\begin{lstlisting}[language=java]
	float []prices = {131.2, 110.8, 117.34};
	float []areas = new float[]{90, 120, 114};
\end{lstlisting}

\remark{
	byte数组内容,不能赋值给int类型的数组,强制转换也不行。
}
\bigskip

以上称之为一位数组。
若提供的房价数据,不止一个地区的,就需要二维数组表示:
houses[地区][下标]。
若需要创建更多维度的数组,以此类推:
patients[性别][年龄][下标]可定位某个病人的状态。
Java数组不要求每一行的长度相等,譬如30岁女性的病人有17个而34岁女性的病人可能是2个或者0个。

\begin{lstlisting}[language=java]
	float [][]houses = {
				{131.2f, 110.8f, 91.09f},
				{157.8f, 107.13f, 105.71f},
				{98.4f, 119.54f, 117.44f},
	};
	int area = 1, index = 2; // 第2地区,第3处房子的价格
	float price23 = prices[area][index];
	System.out.printf("%f", price23); // 输出105.709999
\end{lstlisting}

\remark{
	浮点类型:float和double不是准确数字,只能保证精度。
	因此不要直接和某个浮点数比较是否相等,在允许范围内即可。
}
\bigskip

\begin{exercise}
	double类型的精度是多少?
\end{exercise}

\begin{exercise}
	检查是否与3.14相等的代码?
\end{exercise}

\section{字典}


\section{语句}

\section{函数}