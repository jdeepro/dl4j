% !Mode:: "TeX:UTF-8"
\chapter{Java基础}

\begin{introduction}
	\item 变量和类型
	\item 输入和输出
	\item 面向对象
	\item 异常处理
	\item 多线程
\end{introduction}

相比于其它编程语言,Java的语法不算多的,很容易掌握。
本章将以最短的篇幅带大家掌握Java编程语言。
编程语言三大核心要素:数据、语句和函数。
数据要先存放在变量中,才能被程序处理,不同类型的变量可存放不同的数据;
需要遍历所有的数据的时候,就会用到循环语句;
判断是否成立,使用判断语句等。
把常用的程序逻辑,就会被封装成一个函数。

\section{变量和类型}
与其他静态类型语言一样,java支持的数据类型有:
\lstinline{byte、char、short、int、long、float、double}
等。

\begin{table}[!htbp] \centering \small
\begin{tabular}{|p{1cm}|p{3cm}|p{9cm}|}
\toprule
	\multicolumn{3}{|c|}{基本类型 - 范围}\\
\midrule
	byte&8位(一个字节)&$-128\sim127$\\
	short&16位(两个字节)&$-32768\sim32767$\\
	char&16位(两个字节)&$-32768\sim32767$\\
	int&32位(四个字节)&$-2147483648\sim2147483647$\\
	long&64位(八个字节)&$-9223372036854774808\sim9223372036854774807$\\
	float&32位(四个字节)&$3.402823e+38\sim1.401298e-45$\\
	double&64位(八个字节)&$1.797693e+308\sim4.9000000e-324$\\
\bottomrule
\end{tabular}
	\caption{基本数据类型}
\end{table}

\begin{definition}{变量}{var}
	$[\text{类型}]$ 变量名称, 变量名称 $=$ 初值, $\dots$;\\
	$[\text{类型}]$ 变量名称 $=$ 初值;
\end{definition}

\begin{lstlisting}[language=java]
	int n1, n2=2, n3;
	float f1 = 0.2f;
\end{lstlisting}

\remark{
	变量要在初始化之后,才能使用。代码中小数默认是double类型,需要加后缀f才代表float。
}

\section{语句}

\section{函数}